%%
% This file is part of the User's Guide to RSS
% It contains the appendix for inputfile specifications
%%

\section{Input-file Specifications}

There are exactly four kinds of input files for the \RSS. This includes the project specification files and also the \Lua-script-files that define the robot behaviour.
\begin{enumerate}
	\item The main projectfile containing information about the model. The extension of this type of file is "`.swarm"'.
	\item A file containing robot information. The extension of this file is "`.robot"'.
	\item A file containing obstacle information. The extension of this file is "`.obstacle"'.
	\item \Lua\ file that describes the robot behavouir. The extension of this file is "`.lua"'.
\end{enumerate}


\subsection{Main projectfile}
The following specifications hold only for the main projectfile (with extension \texttt{.swarm}):
\begin{itemize}
	\item A comment begins with a '\#'.
	\item A line is a comment line (beginning with a '\#'), an empty line or a line containing a variable followed by an equal sign followed by a \emph{quoted} value of this variable. Example:
	\begin{verbatim}
		VAR_1="value"
		VAR_2 = "value"
		VAR_3= "value"
		VAR_4 ="value"
	\end{verbatim}
	\item a variable name has to be of the following form: \texttt{[A-Z0-9\_]$^+$}
\end{itemize}


\subsubsection{Variables}
The main project file contains the variables defined in Tables \ref{tab:mainvars} and \ref{tab:mainvars2}.
	
Also the following should be considered:
\begin{itemize}
	\item The order of the variables in the main project file is not important.
	\item If a variable does not appear in the main projectfile, then its default value will be used if such a default value does exist (otherwise an exception will be thrown while loading the main project-file).
\end{itemize}

\clearpage
\begin{sidewaystable}
\scriptsize
	\begin{tabular}{|l|p{0.3\textwidth}|p{0.3\textwidth}|p{0.2\textwidth}|}
		\hline
		\textbf{Variable name} & \textbf{Possible Values} & \textbf{Description} & \textbf{Default}\\\hline\hline
		\texttt{PROJECT\_NAME} & String & Name of the project & -- \\\hline
% 		\texttt{BATTLEBOX\_SIZE} & width, for instance 100 denotes a box of size $100\times 100\times 100$ & Size of bounding box of initial robot positions\\\hline
		\texttt{COMPASS\_MODEL} & Still needs to be specified by the ASG-Team. For instance \texttt{NO\_COMPASS} & Compass model & FULL\_COMPASS\\\hline
		\texttt{ROBOT\_FILENAME} & For instance \texttt{robot\_file}. The extension of the file must not be appended in this variable. & Filename of the robotfile & same as project file\\\hline
		\texttt{OBSTACLE\_FILENAME} & For instance \texttt{obstacle\_file}.  The extension of the file must not be appended in this variable. & Filename of the robotfile & same as project file\\\hline
		\texttt{STATISTICS\_SUBSETS} & A concatenation of none or more of the following strings: \{ALL\}, \{ACTALL\}, \{INACTALL\}, \{MASTERS\}, \{ACTMASTERS\}, \{INACTMASTERS\},  \{SLAVES\}, \{ACTSLAVES\}, \{INACTSLAVES\} &  Defines the subsets of all robots for which to calculate individual statistical data. E.\,g. ``\{ALL\} \{MASTERS\}'' will produce statistical information on \textit{all} robots as well as on \textit{masters only} & NONE\\\hline
		\texttt{STATISTICS\_TEMPLATE} & One of the following: ``ALL'', ``BASIC'' or ``NONE'' & Identifies the set of informations to calculate for each subset. & ALL\\\hline
		\texttt{STATISTICS\_DATADUMP} & Either ``FULL'' or ``NONE'' & Whether or not detailled information (E.\,g. all robots positions at each event) should be streamed to a file during simulation. & NONE\\\hline
		\texttt{ASG} & \texttt{SYNCHRONOUS}, \texttt{ASYNCHRONOUS} or \texttt{SEMISYNCHRONOUS} & Type of ASG & \texttt{SYNCHRONOUS}\\\hline
		  \texttt{ASYNC\_ASG\_SEED} & unsigned int & Seed for asynchronous ASG, only set if ASG=ASYNCHRONOUS & - \\\hline
		    \texttt{ASYNC\_ASG\_PART\_P} & double & Participation Probability for asynch ASG, only set if ASG = ASYNCHRONOUS & - \\\hline
		 \texttt{ASYNC\_ASG\_TIME\_P} & double & parameter governing the timing of asynch ASG, only set if ASG = ASYNCHRNOUS. The lower this is the more often events happen. & - \\\hline
		 
		\texttt{ROBOT\_CONTROL} &  see section \ref{sec:robotControl} & RobotControl to use & -\\\hline
		\texttt{CAMERA\_POSITION} &  \texttt{x,y,z}, where $x,y,z\in\mathbb{R}$& Initial camera position & \texttt{0,0,0}\\\hline
		\texttt{CAMERA\_DIRECTION} &  \texttt{x,y,z}, where $x,y,z\in\mathbb{R}$& Initial camera direction & \texttt{1,0,0}\\\hline


		 
	\end{tabular}
	\caption{Variables in the main project file}\label{tab:mainvars}
\end{sidewaystable}
\thispagestyle{empty}
\clearpage

\clearpage
\begin{sidewaystable}
\scriptsize
	\begin{tabular}{|l|p{0.3\textwidth}|p{0.3\textwidth}|p{0.1\textwidth}|}
		\hline
		\textbf{Variable name} & \textbf{Possible Values} & \textbf{Description} & \textbf{Default}\\\hline\hline

		 \texttt{MARKER\_REQUEST\_HANDLER\_TYPE} &  element from $\{$\texttt{STANDARD,NONE}$\}$ & Type of Marker Request Handler to use & $\{$\texttt{NONE}$\}$\\\hline
		 
		\texttt{TYPE\_CHANGE\_REQUEST\_HANDLER\_TYPE} &  element from $\{$\texttt{STANDARD,NONE}$\}$ & Type of Type Change Request Handler to use. & $\{$\texttt{NONE}$\}$\\\hline
		
		\texttt{POSITION\_REQUEST\_HANDLER\_TYPE} &  element from $\{$\texttt{VECTOR,NONE}$\}$ & Type of Position Request Handler to use & $\{$\texttt{NONE}$\}$\\\hline

		\texttt{VELOCITY\_REQUEST\_HANDLER\_TYPE} &  element from $\{$\texttt{VECTOR,NONE}$\}$ & Type of Velocity Request Handler to use & $\{$\texttt{NONE}$\}$\\\hline

		\texttt{ACCELERATION\_REQUEST\_HANDLER\_TYPE} &  element from $\{$\texttt{VECTOR,NONE}$\}$ & Type of Acceleration Request Handler to use & v\\\hline
		
		 \texttt{STANDARD\_MARKER\_REQUEST\_HANDLER\_SEED} &  integer & Seed for Marker Request Handler to use & $\{$\texttt{NONE}$\}$\\\hline
		 
		\texttt{STANDARD\_TYPE\_CHANGE\_REQUEST\_HANDLER\_SEED} &   integer & Seed for Type Change Request Handler to use. & -\\\hline
		
		\texttt{POSITION\_REQUEST\_HANDLER\_SEED} &   integer & Seed for Position Request Handler to use & -\\\hline

		\texttt{VELOCITY\_REQUEST\_HANDLER\_SEED} &   integer & Seed for Velocity Request Handler to use & -\\\hline

		\texttt{ACCELERATION\_REQUEST\_HANDLER\_SEED} &   integer & Seed for Acceleration Request Handler to use & -\\\hline
		
		
		
		 \texttt{STANDARD\_MARKER\_REQUEST\_HANDLER\_DISCARD\_PROB} &  element from interval $[0,1]$ & Discard probability for Marker Request Handler to use & -\\\hline
		 
		\texttt{STANDARD\_TYPE\_CHANGE\_REQUEST\_HANDLER\_DISCARD\_PROB} & element from interval $[0,1]$ & Discard probability  for Type Change Request Handler to use. & -\\\hline
		
		\texttt{POSITION\_REQUEST\_HANDLER\_DISCARD\_PROB} & element from interval $[0,1]$ & Discard probability  for Position Request Handler to use & -\\\hline

		\texttt{VELOCITY\_REQUEST\_HANDLER\_DISCARD\_PROB} & element from interval $[0,1]$ & Discard probability  for Velocity Request Handler to use & -\\\hline

		\texttt{ACCELERATION\_REQUEST\_HANDLER\_DISCARD\_PROB} & element from interval $[0,1]$ & Discard probability  for Acceleration Request Handler to use & -\\\hline
		
		
		\texttt{POSITION\_REQUEST\_HANDLER\_MODIFIER} & list of vector modifiers (see \ref{sec:vectorModifiers}) & List of vector modifiers for Position Request Handler to use & -\\\hline

		\texttt{VELOCITY\_REQUEST\_HANDLER\_MODIFIER} & list of vector modifiers (see \ref{sec:vectorModifiers}) & List of vector modifiers for Velocity Request Handler to use & -\\\hline

		\texttt{ACCELERATION\_REQUEST\_HANDLER\_MODIFIER} & list of vector modifiers (see \ref{sec:vectorModifiers}) & List of vector modifiers for Acceleration Request Handler to use & -\\\hline		
				
	\end{tabular}
	\caption{Variables in the main project file}\label{tab:mainvars2}
\end{sidewaystable}
\enlargethispage*{2cm}
\thispagestyle{empty}
\clearpage


\subsubsection{Example of a main project file}
A main project file may look like:
\lstset{language=tcl}
\begin{lstlisting}
# 
# Description about configuration.
#
	
	PROJECT_NAME="My Exciting Project"
	COMPASS_MODEL="NO_COMPASS"
	ROBOT_FILENAME="myrobots"
	OBSTACLE_FILENAME="myobstacle"
	STATISTICS_MODULE="0"
	ASG="ASYNCHRONOUS"
	ROBOT_CONTROL="ROBOT_TYPE_ROBOT_CONTROL"
	MASTER_VIEW="GLOBAL_VIEW"
	SLAVE_VIEW="ONE_POINT_FORMATION_VIEW"
	SLAVE_ONE_POINT_FORMATION_VIEW_RADIUS="5.0"
	
	CAMERA_POSITION="0,0,0"
	CAMERA_DIRECTION="1.5,0,0.5"
	
	MARKER_REQUEST_HANDLER_TYPE="STANDARD"
	STANDARD_MARKER_REQUEST_HANDLER_DISCARD_PROB="0.5"
	STANDARD_MARKER_REQUEST_HANDLER_SEED="1"

	TYPE_CHANGE_REQUEST_HANDLER_TYPE="NONE"
	# no additional variables needed

	POSITION_REQUEST_HANDLER_TYPE="VECTOR"
	VECTOR_POSITION_REQUEST_HANDLER_DISCARD_PROB="0.1"
	VECTOR_POSITION_REQUEST_HANDLER_SEED="3"
	VECTOR_POSITION_REQUEST_HANDLER_MODIFIER="(VECTOR_TRIMMER,1.5);(VECTOR_RANDOMIZER,5,2.5)"

	VELOCITY_REQUEST_HANDLER_TYPE="VECTOR"
	VECTOR_VELOCITY_REQUEST_HANDLER_DISCARD_PROB="0.1"
	VECTOR_VELOCITY_REQUEST_HANDLER_SEED="3"
	VECTOR_VELOCITY_REQUEST_HANDLER_MODIFIER="(VECTOR_TRIMMER,1.5);(VECTOR_RANDOMIZER,5,2.5)"
\end{lstlisting}


\subsection{Robot file}
The robotfile uses a csv-compatible format.
Therefore the information for one robot has to be saved in exactly one line of the file.
Each line contains the following data. The order of this data is important!

You can declare specific lines as comments by setting \# as the first sign of
the corresponding line.
\begin{itemize}
	\item ID-number
	\item initial position ($x,y,z$)
	\item initial type (for instance master, slave,$\ldots$)
	\item initial velocity ($x,y,z$)
	\item initial acceleration ($x,y,z$)
	\item initial status (maybe sleeping or ready; still has to be specified more precisely)
	\item initial marker information (still has to be specified)
	\item algorithm to use (shortcut for an algorithm; still needs to be specified)
	\item color (using this color a robot is marked for instance for a special treatment during the visualization; this color isn't used anywhere else); Integers correspond to the following colors: 0 green, 1 blue, 2 cyan, 3 red, 4 magenta, 5 yellow, 6 white, 7 black, 8 orange, 9 purple
	\item coordinate system axes (triple $x_1,x_2,x_3,y_1,y_2,y_3,z_1,z_2,z_3$; this field will be left empty, if axes are supposed to be generated uniformly at random)
\end{itemize}
The first line always is (column headers):
\begin{lstlisting}
	  "ID","x-position","y-position","z-position","type","x-velocity","y-velocity","z-velocity","x-acceleration","y-acceleration","z-acceleration","status","marker-info","algorithm","color","x-axis-1","x-axis-2","x-axis-3","y-axis-1","y-axis-2","y-axis-3","z-axis-1","z-axis-2","z-axis-3"
\end{lstlisting}
Each non-number is quoted and each number may be quoted.

\subsubsection{Example of a robot file}
\begin{lstlisting}
	"ID","x-position","y-position","z-position","type","x-velocity","y-velocity","z-velocity","x-acceleration","y-acceleration","z-acceleration","status","marker-info","algorithm","color","x-axis-1","x-axis-2","x-axis-3","y-axis-1","y-axis-2","y-axis-3","z-axis-1","z-axis-2","z-axis-3"
	0,5.3,9.2,6.4,"master",1.5,2.5,3.5,1.5,2.5,3.5,"sleeping",0,0,0,1,0,0,0,1,0,0,0,1
	1,"2.5","4.2","8.8","slave",1.5,2.5,3.5,1.5,2.5,3.5,"ready",0,1,0,1,0,0,0,1,0,0,0,1
\end{lstlisting}

\subsection{Obstacle file}
Like the robot file the obstacle file uses a csv-compatible format. 
Therefore the information for one robot has to be saved in exactly one line of the file.
Each line contains the following data. The order of this data is important!

You can declare specific lines as comments by setting \# as the first sign of
the corresponding line.
\begin{itemize}
	\item type (marker, sphere or box)
	\item position $(x,y,z)$
	\item marker information (still needs to be specified)
	\item $x/y/z$-lengths or radius (depending on type)
\end{itemize}

The first line always is (column headers):
\begin{lstlisting}
"type","x-position","y-position","z-position","marker-info","size-info","",""
\end{lstlisting}
Each non-number is quoted.


\subsubsection{Example of an obstacle file}
\begin{lstlisting}
"type","x-position","y-position","z-position","marker-info","size-info","",""
"box",2.0,3.0,4.0,0,1.0,2.0,3.0,
"sphere",3.4,5.2,5.1,0,5.0,"",""
"marker",3.5,1.4,5.1,0,"","",""
\end{lstlisting}
As you can already see in the example, if the type of an obstacle is sphere, then the last two values must be empty, i.\,e. '',''. Analoguos, if the type is marker, the last three values must be empty, i.\,e. '','',''.
