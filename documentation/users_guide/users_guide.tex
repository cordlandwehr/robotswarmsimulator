\documentclass[a4paper,halfparskip,11pt,twoside]{scrartcl}
\usepackage{schlaue_schwaerme}
\usepackage{url}

\pdfinfo{
	/Title   (RobotSwarmSimulator User's Guide)
	/Subject (User's Guide)
	/Author  (A. Klaas, A. Cord-Landwehr, C. Raupach, C. Weddemann, D. Warner, D. Wonisch, K. Swierkot, M. Märtens, M. Hüllmann, P. Kling, S. Kurras)
	/Keywords (Robot, Swarm, Simulator, Paderborn, University)
}


\newcommand{\R}{\mathbb{R}}
\newcommand{\cupdot}{\ensuremath{\mathaccent\cdot\cup}} % disjuncted union
%TODO(cola) move to swarm-package
\newcommand{\gnuplot}{ {\sffamily gnuplot}\xspace}

\lstdefinelanguage{lua}
{morekeywords={and, break, do, else, elseif, end, false, for, function, if, in, local, nil, not, or, repeat, return, then, true, until, while, print},
sensitive=false,
morecomment=[l]{--},
morestring=[bd]",
morestring=[bd]',
%morestring=[s]{[[}{]]},
}

\begin{document}
\init{}

\pgMaketitle{User's Guide}{
\begin{pgArrowitemize}
	\item Alexander Klaas
	\item Andreas Cord-Landwehr
	\item Christoph Raupach
	\item Christoph Weddemann
	\item Daniel Warner
	\item Daniel Wonisch
	\item Kamil Swierkot
	\item Marcus M\"artens
	\item Martina H\"ullmann
	\item Peter Kling
	\item Sven Kurras
\end{pgArrowitemize}
}

\section{Introduction}
This document shall help you to use the \RSS\ to simulate robot swarms and to get useful information by the visualization and statistics modules.

For this purpose the document is divided in several sections. It will be useful if you just start with the ``Getting Started'' section and play a little bit with the simulator and continue by reading the more detailed information on input parameters, input file specifications, generating simulations and getting statistics.


\section{Getting Started}
For a first test of \RSS you can simply run it in generation mode, and use the generated scenario with one of our default algorithms. All example files can be found in subdirectory {\tt ProjectFiles} of the install directory. Here we will present a short example on how to simulate the behavior of the {\sffamily Center of Gravity} algorithm (COG).

\begin{enumerate}
	\item Change to the directory that contains the \RSS binary.
	\item Run the following command:

		\texttt{RobotSwarmSimulator -{}-generate -{}-distr-pos 20 -{}-add-pos-handler \textbackslash\textbackslash
			\linebreak -{}-robots 1001 -{}-algorithm COGRobot}

		This will generate a simulation specification in form of the following files:
		\begin{itemize}
			\item {\tt newrandom.swarm} -- contains information about the simulation process
			\item {\tt newrandom.robot} -- contains information about the robots
			\item {\tt newrandom.obstacle} -- contains information about the obstacles
		\end{itemize}
		The last two files are referenced in the {\tt .swarm} file. Thus, to load the simulation, you only need to load the {\tt .swarm} file. Note that all these files
are simple text files that can be edited by hand.
	\item Run the command:

		\centerline{\tt RobotSwarmSimulator -{}-project-file newrandom -{}-output mylogs}

		This will start the simulation process. There are various keyboard shortcuts that can be used to control the simulation. Press \fbox{\tt h} for an overview. You can quit the simulation by pressing \fbox{\tt q}.
	\item The previous step will generate various output files in the subdirectory {\tt mylogs}, mainly produced by the statistics module of the \RSS. You may directly use \gnuplot to analyze these files.
\end{enumerate}
This simple example can be used as a base for further test runs. Please look at the following sections to get specifications of the input files and the user inferface.

\section{Running the \RSS\ with Parameters}
Each execution of the \RSS needs specific parameters that are mandatory. By running the \RSS\ you have to specify at least, which kind of execution you want. There are two main options:

\begin{itemize}
	\item By adding the parameter {\tt -{}-generate} the generation mode is started to create new input files. This will generate three different files having the suffixes {\tt .swarm}, {\tt .robot} and {\tt .obstacle}. The most important of these is the {\tt .swarm} file, which references the other two files.
	\item Using the parameter {\tt -{}-project-file $\langle$your\_input\_file$\rangle$}, the simulation mode is started. In this case, \texttt{<your\_input\_file>} should be the name of a file having the suffix {\tt .swarm} (e.\,g. generated with generation mode).
\end{itemize}

There are other command line options that may be used to show help or about messages and to customize the generation or simulation mode. Start the \RSS with the parameter {\tt -{}-help} to get an overview. A typical session may look like this:

\begin{verbatim}
   ./RobotSwarmSimulator --help
   ./RobotSwarmSimulator --project-file <path_to_testdata>/testfile_2 \\
                         --output <path_to_output_files> \\
                         --history-length 10
\end{verbatim}

All options listed in Listing~\ref{lst:RSS-help} can be used. Further information for this parameters can be found in the following sections. The definition of most of the parameters can be found in Table~\ref{tab:mainvars}.

\begin{lstlisting}[caption={\RSS Helpline},label=lst:RSS-help]
localhost:~$ ./RobotSwarmSimulator --help

General options:
 --help                shows this help message
 --version             shows version of RobotSwarmSimulator
 --about               tells you who developed this awesome piece of software

Generator options:
 --generate                      switch to generator mode
 --seed arg (=1)                 seed for random number generator
 --robots arg (=100)             number of robots
 --algorithm arg (=NONE)         name of algorithm or lua-file
 --swarmfile arg (=newrandom)    swarm-file for output
 --robotfile arg (=newrandom)    robot-file for output
 --obstaclefile arg (=newrandom) obstacle-file for output
 --add-pos-handler               add position request handler for testing
 --add-vel-handler               add velocity request handler for testing
 --add-acc-handler               add acceleration request handler for testing
 --distr-pos arg (=0)            distribute position in cube [0;distr-pos]^3
 --min-vel arg (=0)              distribute velocity in sphere with minimal  
                                 absolut value min-vel
 --max-vel arg (=0)              distribute velocity in sphere with maximal
                                 absolute value max-vel
 --min-acc arg (=0)              distribute acceleration in sphere with minimal  
                                 absolut value min-acc
 --max-acc arg (=0)              distribute acceleration in sphere with maximal
                                 absolute value max-acc
 --distr-coord                   distribute robot coordinate-systems uniformly

Simulation options:
 --project-file arg         Project file to load
 --output arg               Path to directory for output
 --history-length arg (=25) history length
 --dry                      disables statistics output
 --blind                    disables visual output
 --steps                   if set this terminates the simulation after a given amount of steps
\end{lstlisting}


\subsection{General options}
\begin{description}
	\item [-{}-help] Lists all possible options including a short description.
	\item [-{}-version] This option shows the version information of your \RSS.
	\item [-{}-about] Get information about the developer team, contact information and more.
\end{description}

\subsection{Generator options}
\begin{description}
	\item [-{}-generate] Switch to generator mode. This is necessary for the further options of this section.
	\item [-{}-seed arg] Sets the seed for the random number generator for robot generation. If not set the seed is {\tt 1}. An unsigned integer value is expected.
	\item [-{}-robots arg] The number of robots to be generated. The default number is 100. An unsigned integer value is expected.
	\item [-{}-algorithm arg] The name of the algorithm the robots should use. If not set the algorithm {\tt SimpleRobot} is used. This is only a stub without any functionality. Also the name of a \Lua-file can be given. The extension {\tt .lua} is mandatory for lua-files.
	\item [-{}-swarmfile arg] The name of the swarmfile that shall be generated. Default is {\tt newrandom}. Filename without extension is expected.
	\item [-{}-robotfile arg] The name of the robotfile that shall be generated. Default is {\tt newrandom}. Filename without extension is expected.
	\item [-{}-obstaclefile arg] The name of the obstaclefile that shall be generated. Default is {\tt newrandom}. Filename without extension is expected.
	\item [-{}-add-pos-handler] Causes the generated files to contain a position request handler with reasonable default values. If you need a more sophisticated position request handler, you have to edit the generated {\tt .swarm} file yourself.
	\item [-{}-add-vel-handler] Causes the generated files to contain a velocity request handler with reasonable default values. If you need a more sophisticated velocity request handler, you have to edit the generated {\tt .swarm} file yourself.
	\item [-{}-add-acc-handler] Causes the generated files to contain a acceleration request handler with reasonable default values. If you need a more acceleration request handler, you have to edit the generated {\tt .swarm} file yourself.
	\item [-{}-distr-pos arg] Distributes the position of robots uniformly at random in the cube $[-arg/2,+arg/2]^3$
 If not set, all robots are at position zero.
	\item [-{}-min-vel arg] The robots will be generated with a velocity distributed uniformly in a sphere with the given minimum and maximum absolute value (see {\tt -{}---max-vel}). This parameter defaults to 0.
	\item [-{}-max-vel arg] The robots will be generated with a velocity distributed uniformly in a sphere with the given minimum and maximum absolute value (see {\tt -{}---min-vel}). This parameter defaults to 0.
	\item [-{}-min-acc arg] The robots will be generated with a acceleration distributed uniformly in a sphere with the given minimum and maximum absolute value (see {\tt -{}-max-acc}). This parameter defaults to 0.
	\item [-{}-max-acc arg] The robots will be generated with a acceleration distributed uniformly in a sphere with the given minimum and maximum absolute value (see {\tt -{}-min-acc}). This parameter defaults to 0.
	\item [-{}-distr-coord arg] Generates uniformly distributed coordinate--systems for the robots. If this option is not given, all robots will have the same global coordinate system (defined by a unit matrix).
 If not set, all velocities are zero.
\end{description}

\subsection{Simulation options}
\begin{description}
	\item [-{}-project-file arg] Specifies the project file use. Use this parameter to load a simulation specified by a {\tt .swarm} file. Not that you may ommit the file extension {\tt .swarm}. Mandatory for simulation.
	\item [-{}-output arg] Specifies a directory to be used for files generated by the simulation (e.\,g. statistics files for \gnuplot). The given name is interpreted relative to the current directory and will create the directory if necessary. If ommitted, all generated files will be stored in the current directory.
	\item [-{}-history-length arg] Sets the history length, i.\,e. the length of the ringbuffer that stores past simulation states. Standard is 25 (which should be a good choice in most cases). An unsigned integer is expected.
	\item [-{}-dry] No statistic files are beeing generated.
\end{description}


\section{Using the Simulator-Interface}
During the simulation it is possible to interact with the simulation in different ways. The following hot-keys are supported while simulating:

\begin{description}
	\item [\fbox{\tt Space}] Start/ Stop.
	\item [\fbox{\tt q}] Quit the \RSS.
	\item [\fbox{\tt F1}] Help!
	\item [\fbox{\tt g}] Show the center of gravity of the swarm.
	\item [\fbox{\tt v}] Show velocity vectors.
	\item [\fbox{\tt b}] Show acceleration vectors.
	\item [\fbox{\tt k}] Show global coordinates system.
	\item [\fbox{\tt w},\fbox{\tt s}] In the corresponding camera mode use \fbox{\tt w} for up  and \fbox{\tt s} for down.
	\item [Arrow-Keys] Moves the view: left, right, before, behind.
	\item [\fbox{\tt m}] Use \fbox{\tt m} to switch to mouse spinning mode and use your mouse to rotate the view. Note that this is not supported in every camera mode.
	\item [\fbox{\tt +}, \fbox{\tt -}] Increase/ decrease simulation-speed by constant.
	\item [\fbox{\tt *},\fbox{\tt /}] Double/ half simulation-speed.
	\item [\fbox{\tt c}] Change camera mode.
\end{description}

\subsection{Information from Vizualisation}
%TODO (cola) at this point ther should be an diagram with size relationships of the different objects, axis, robots etc)
At the beginning of a simulation the camera view is directed to point \texttt{(0,0,0)}, if not explicitely specified. The axis, displayed when activated by pressing \fbox{\tt K} are scaled with each unit equals to \texttt{2}. All presented robots always have diameter \texttt{0.15}, where the ball is defined with center equals to the roboter position.



\section{Create Robot Algorithms for the \RSS}
There are two ways to define a new robot. One way is to write a subclass of \texttt{Robot} and add a new condition in \texttt{factories.cc}. The other way is to define the robot algorithm by the {\sffamily Lua} scripting language and to load the algorithm at run-time. We want to stress, that a definition of robot algorithms by \Lua scripts only scales for small numbers of robots. Thus, for robot swarms of sizes greater than 500 robots you will (on standard computers) recognize a lack of performance.

\subsection{Create Robot Algorithms by Lua Scripts}
For information on how to write \Lua scripts please visit \url{http://www.lua.org} and use the documentation presented there. For interacting the environment \Lua scripts may access the following functions and constants (if allowed by the current view and if the according request handlers are set):

\paragraph{Lua Functions}
%TODO(cola) document ALL of these functions
\begin{description}
	\item [\texttt{get\_visible\_robots()}] Returns the array of visible robots.
	\item [\texttt{get\_visible\_obstacles()}] Returns the array of visible obstacles.
	\item [\texttt{get\_visible\_markers()}] Returns the array of visible markers.
	\item [\texttt{get\_position(<robot>)}] Returns the position of the calling robot as a Vector3d.
	\item [\texttt{get\_marker\_information(<robot>)}] Returns the MarkerInformation of the calling\linebreak robot.
	\item [\texttt{get\_id(<robot>)}] Returns the identifier of the calling robot.
	\item [\texttt{get\_robot\_acceleration(<robot>)}] Returns the acceleration of the calling robot as Vector3d.
	\item [\texttt{get\_robot\_coordinate\_system\_axis(<robot>)}] Returns the coordinate system of the calling robot as a CoordinateSystem.
	\item [\texttt{get\_robot\_type(<robot>)}] Returns the type of the calling robot as a RobotType.
	\item [\texttt{get\_robot\_status(<robot>)}] Returns the status of the calling robot as a RobotStatus.
	\item [\texttt{is\_point\_in\_obstacle(<obstacle>, <point>)}] Returns true iff the given point of\linebreak type Vector3d is within the given obstacle.
	\item [\texttt{get\_box\_depth(<box>)}] Returns the depth of the given box as a double.
	\item [\texttt{get\_box\_width(<box>)}] Returns the width of the given box as a double.
	\item [\texttt{get\_box\_height(<box>)}] Returns the height of the given box as a double.
	\item [\texttt{get\_sphere\_radius(<sphere>)}] Returns the radius of the given sphere as a double.
	\item [\texttt{is\_box\_identifier(<identifier>)}] Returns true iff the given identifier is an identifier of a box.
	\item [\texttt{is\_sphere\_identifier(<identifier>)}] Returns true iff the given identifier is an identifier of a sphere.
	\item [\texttt{add\_acceleration\_request(<Vector3d>)}] 
	\item [\texttt{add\_position\_request(<Vector3d>)}]
	\item [\texttt{add\_velocity\_request(<Vector3d>)}]
	\item [\texttt{add\_marker\_request(<marker>)}]
	\item [\texttt{add\_type\_change\_request(<type>)}]
	\item [\texttt{get\_own\_identifier()}]
\end{description}

\paragraph{Geometry Package}
\begin{description}
	\item [\texttt{is\_in\_smallest\_bbox(<vector of Vector3d>,<Vector3d>)}]
	\item [\texttt{compute\_distance(<Vector3d, Vector3d>)}]
	\item [\texttt{compute\_cog(<vector of Vector3d>)}]
	\item [\texttt{sort\_vectors\_by\_length(<vector of Vector3d>)}] Sorts Vector3d points by distance to origin
\end{description}


\paragraph{\Lua Constants}
\begin{description}
	\item [\texttt{RobotType}] \texttt{SLAVE, MASTER}
	\item [\texttt{RobotStatus}] \texttt{SLEEPING, READY}
\end{description}

\paragraph{Special Variable Types}
\begin{description}
	\item [\texttt{Vector3d}] This type is the \Lua equivalent to Vector3d in \RSS
	\begin{itemize}
		\item Operators: \texttt{+, *, /}
		\item Dimensions: \texttt{x, y, z}
	\end{itemize}
	\item [\texttt{MarkerInformation}] This is the data type for marker information. The information can be accessed by the according operators:
	\begin{itemize}
		\item Operators: \texttt{add\_data, get\_data}
	\end{itemize}
	\item [\texttt{CoordinateSystem}] This type consists of three Vector3d objects. The axes can be accessed by the following methods:
	\begin{itemize}
		\item Operators: \texttt{x\_axis, y\_axis, z\_axis}
	\end{itemize}

\end{description}

\subsubsection{Example Algorithm}
Listing~\ref{lst:cog-lua} shows you how to formulate the COG-algorithm in \Lua.

\lstset{language=lua}
\begin{lstlisting}[caption={COG algorithm in \Lua},label=lst:cog-lua]
function main() 
    robots = get_visible_robots();
    center = get_position(get_own_identifier());
    for i = 1, #robots do
        center = center + get_position(robots[i]);
    end
    center = center / (#robots+1);
    add_position_request(center);
end
\end{lstlisting}

\subsection{Create Robot Algorithms in C++}
For creating a new robot algorithm inside the simulator you need to do the following steps:
\begin{enumerate}
	\item Create a subclass of \texttt{Robot}.
	\item Put the algorithm into \texttt{src/RobotImplementations/}.
	\item Write the method \texttt{compute()}.
	\item Write the method \texttt{get\_algorithm\_id()}.
	\item Add the algorithm identifier as option in file\newline \texttt{src/SimulationKernel/factories.cc} in method \texttt{robot\_factory(\dots)}.
\end{enumerate}



\section{Statistics}
A simulation run results in three output files of statistic data:

\begin{itemize}
\item \texttt{gnuplot\_20091224\_184129\_ALL.plt} (GNUPlot-configuration file)
\item \texttt{output\_20091224\_184129\_ALL.plt} (according statistic data)
\item \texttt{output\_20091224\_184129\_DATADUMP\_FULL.plt} (complete data dump)
\end{itemize}

The filenames results from current date (year, month, day), the current time (hour, minute, second), followed by description of observed object subset (e.\,g. \texttt{ALL}, \linebreak \texttt{MASTERS,}\dots).

\newpage
\appendix

\section{Visualization with \gnuplot}

\paragraph{Display with \gnuplot}
To display statistical information about a finished simulation run the correct \gnuplot configuration file needs to be opened with \gnuplot. On Unix systems (with installed \gnuplot) the interactive \gnuplot shell can be used ( (\texttt{> load }``\texttt{gnuplotfile.plt}''). More commands can be displayed using \texttt{> help}. Also it is possible to directly display the file with \texttt{\$ gnuplot -persist }``\texttt{gnuplotfile.plt}''. The additional option \texttt{-persist} causes the window to stay open after the file has been opened. For more informaion on \gnuplot read the manual page \texttt{\$ man gnuplot}.


\paragraph{Display configuration}
A file  \texttt{\scriptsize output\_20091224\_184129\_ALL.plt} might look like this:

\begin{center}
\texttt{\begin{tabular}{rrrl}
\# time & avg\_spd & minball\_x & (\dots) \\
\hline
0 & 9.2 & 20.5 \\
1 & 11.3 & 21.5 \\
4 & 9.1 & 21.9 \\
6 & 7.5 & 22.3
\end{tabular}}
\end{center}

Each line contains extremly valuable statistical data for a certain time. The start of a new column is marked by one (or more) whitespaces.
comment lines are started with a '\#' sign.

The file  \texttt{gnuplot\_20091224\_184129\_ALL.plt} also contains the formatting of the extremly valuable statistical data. The formatting can be changed if so desired. 

\begin{verbatim}
# statistics of the simulation
#====================================
set title ' SCHLAUE SCHWÄRME '
set xrange []
set yrange []
set grid
set pointsize 0.5
set xlabel 'time'
set ylabel ''
plot 'output_(...)_ALL.plt' using 1:2 title 'avg_spd' with linespoints,\
     'output_(...)_ALL.plt' using 1:3 title 'minball_x' with linespoints
\end{verbatim}

The lines have the following meaning:
\begin{itemize}
\item \texttt{set title '\textit{Titel}'} generates a title within the upper part of the graphical display.
\item \texttt{set xrange [\textit{min}:\textit{max}]} limits the visible area (horizontal) to the area between \texttt{\textit{min}} ans \texttt{\textit{max}}, \texttt{set xrange []} scales the extremly valuable statistical data based on the datapoints.
\item \texttt{set grid} displays a beautiful grid.
\item \texttt{set pointsize \textit{multiplikator}} scales the size of the points (if there are any) according to \texttt{\textit{multiplikator}}.
\item \texttt{xlabel '\textit{Label}'} labels the x-axis with useful information.
\end{itemize}
These choices only concern the graphical display. The configuration of the displayed data is done later.

\begin{itemize}
 \item \texttt{plot '\textit{Dateiname.plt}'} starts the display process. Data is read from the input file. The parameter \texttt{using 1:2} uses the first column as x-axis and the second column as y-axis. \texttt{title `avg\_speed'} adds a title for the function defined by this. The addition \texttt{with linespoints} causes the data points to be displayed with lines between them. \texttt{linespoints} can be replaced by:
\begin{itemize}
\item \texttt{lines} connects each data point with a line
\item \texttt{dots}displays datapoints as dots (for many datapoints)
\item \texttt{points} displays datapoints as points
\item \texttt{linespoints} displays a combination of \textit{points} and \textit{lines}
\item \texttt{impulses} displays a vertical line for each data point
\end{itemize}

\end{itemize}

\begin{figure}[p]
	\begin{center}
	\includegraphics[width=0.9\textwidth]{stats-howto-gnuplot.png}
	\caption{Example result output from \gnuplot}
	\end{center}
\end{figure}


\paragraph{More about \gnuplot}~\\

\begin{tabular}{l}
 $[1]$ Homepage: \texttt{\scriptsize http://www.gnuplot.info/} \\
 $[2]$ Introduction course: \texttt{\scriptsize http://userpage.fu-berlin.de/\~voelker/gnuplotkurs/gnuplotkurs.html}
\end{tabular}

\newpage 
%%
% This file is part of the User's Guide to RSS
% It contains the appendix for inputfile specifications
%%
\label{cp:inputfile}
\section{Input-file Specifications}

There are exactly four kinds of input files for the \RSS. This includes the project specification files and also the \Lua-script-files that define the robot behavior.
\begin{enumerate}
	\item The main projectfile containing information about the model. The extension of this type of file is ``.swarm''.
	\item A file containing robot information. The extension of this file is ``.robot''.
	\item A file containing obstacle information. The extension of this file is ``.obstacle''.
	\item \Lua\ file that describes the robot behavouir. The extension of this file is ``.lua''.
\end{enumerate}
For easier handling of the robot and obstacle files it is also possible to take the ending ``.csv''. In this case the ending also must be part of the filename in the main projedtfile.

\subsection{Main projectfile}
The following specifications hold only for the main projectfile (with extension \texttt{.swarm}):
\begin{itemize}
	\item A comment begins with a '\#'.
	\item A line is a comment line (beginning with a '\#'), an empty line or a line containing a variable followed by an equal sign followed by a \emph{quoted} value of this variable. Example:
	\begin{verbatim}
		VAR_1="value"
		VAR_2 = "value"
		VAR_3= "value"
		VAR_4 ="value"
	\end{verbatim}
	\item a variable name has to be of the following form: \texttt{[A-Z0-9\_]$^+$}
\end{itemize}


\subsubsection{Variables}
The main project file contains the variables defined in Tables~\ref{tab:mainvars} and \ref{tab:mainvars2}.
	
Also the following should be considered:
\begin{itemize}
	\item The order of the variables in the main project file is not important.
	\item If a variable does not appear in the main projectfile, then its default value will be used if such a default value does exist (otherwise an exception will be thrown while loading the main project-file).
\end{itemize}

\clearpage
\begin{sidewaystable}
\scriptsize
	\begin{tabular}{|l|p{0.3\textwidth}|p{0.3\textwidth}|p{0.2\textwidth}|}
		\hline
		\textbf{Variable name} & \textbf{Possible Values} & \textbf{Description} & \textbf{Default}\\\hline\hline
		\texttt{PROJECT\_NAME} & String & Name of the project & -- \\\hline
% 		\texttt{BATTLEBOX\_SIZE} & width, for instance 100 denotes a box of size $100\times 100\times 100$ & Size of bounding box of initial robot positions\\\hline
		\texttt{COMPASS\_MODEL} & Still needs to be specified by the ASG-Team. For instance \texttt{NO\_COMPASS} & Compass model & FULL\_COMPASS\\\hline
		\texttt{ROBOT\_FILENAME} & For instance \texttt{robot\_file}. The extension of the file must not be appended in this variable. & Filename of the robotfile & same as project file\\\hline
		\texttt{OBSTACLE\_FILENAME} & For instance \texttt{obstacle\_file}.  The extension of the file must not be appended in this variable. & Filename of the robotfile & same as project file\\\hline
		\texttt{STATISTICS\_SUBSETS} & A concatenation of none or more of the following strings: \{ALL\}, \{ACTALL\}, \{INACTALL\}, \{MASTERS\}, \{ACTMASTERS\}, \{INACTMASTERS\},  \{SLAVES\}, \{ACTSLAVES\}, \{INACTSLAVES\} &  Defines the subsets of all robots for which to calculate individual statistical data. E.\,g. ``\{ALL\} \{MASTERS\}'' will produce statistical information on \textit{all} robots as well as on \textit{masters only} & NONE\\\hline
		\texttt{STATISTICS\_TEMPLATE} & One of the following: ``ALL'', ``BASIC'' or ``NONE'' & Identifies the set of informations to calculate for each subset. & ALL\\\hline
		\texttt{STATISTICS\_DATADUMP} & Either ``FULL'' or ``NONE'' & Whether or not detailled information (E.\,g. all robots positions at each event) should be streamed to a file during simulation. & NONE\\\hline
		\texttt{ASG} & \texttt{SYNCHRONOUS}, \texttt{ASYNCHRONOUS} or \texttt{SEMISYNCHRONOUS} & Type of ASG & \texttt{SYNCHRONOUS}\\\hline
		  \texttt{ASYNC\_ASG\_SEED} & unsigned int & Seed for asynchronous ASG, only set if ASG=ASYNCHRONOUS & - \\\hline
		    \texttt{ASYNC\_ASG\_PART\_P} & double & Participation Probability for asynch ASG, only set if ASG = ASYNCHRONOUS & - \\\hline
		 \texttt{ASYNC\_ASG\_TIME\_P} & double & parameter governing the timing of asynch ASG, only set if ASG = ASYNCHRNOUS. The lower this is the more often events happen. & - \\\hline
		 
		\texttt{ROBOT\_CONTROL} &  see section \ref{sec:robotControl} & RobotControl to use & -\\\hline
		\texttt{CAMERA\_POSITION} &  \texttt{x,y,z}, where $x,y,z\in\mathbb{R}$& Initial camera position of startup Camera & \texttt{0,0,0}\\\hline
		\texttt{CAMERA\_DIRECTION} &  \texttt{x,y,z}, where $x,y,z\in\mathbb{R}$& Initial camera direction of startup camera & \texttt{1,0,0}\\\hline
		\texttt{CAMERA\_TYPE} &  $[$"FOLLOW"|"FREE"|"COG"$]$& Sets the startup camera& FOLLOW\\\hline
		\texttt{STATISTICS\_FILEID} & string & ID for simulation output, is used instead of timestep\\\hline
	\end{tabular}
	\caption{Variables in the main project file}\label{tab:mainvars}
\end{sidewaystable}
\thispagestyle{empty}
\index{ASG}\index{View}
\clearpage

\clearpage
\begin{sidewaystable}
\scriptsize
	\begin{tabular}{|l|p{0.3\textwidth}|p{0.3\textwidth}|p{0.1\textwidth}|}
		\hline
		\textbf{Variable name} & \textbf{Possible Values} & \textbf{Description} & \textbf{Default}\\\hline\hline

		 \texttt{MARKER\_REQUEST\_HANDLER\_TYPE} &  element from $\{$\texttt{STANDARD,NONE}$\}$ & Type of Marker Request Handler to use & $\{$\texttt{NONE}$\}$\\\hline
		 
		\texttt{TYPE\_CHANGE\_REQUEST\_HANDLER\_TYPE} &  element from $\{$\texttt{STANDARD,NONE}$\}$ & Type of Type Change Request Handler to use. & $\{$\texttt{NONE}$\}$\\\hline
		
		\texttt{POSITION\_REQUEST\_HANDLER\_TYPE} &  element from $\{$\texttt{VECTOR,COLLISION,NONE}$\}$ & Type of Position Request Handler to use & $\{$\texttt{NONE}$\}$\\\hline

		\texttt{VELOCITY\_REQUEST\_HANDLER\_TYPE} &  element from $\{$\texttt{VECTOR,NONE}$\}$ & Type of Velocity Request Handler to use & $\{$\texttt{NONE}$\}$\\\hline

		\texttt{ACCELERATION\_REQUEST\_HANDLER\_TYPE} &  element from $\{$\texttt{VECTOR,NONE}$\}$ & Type of Acceleration Request Handler to use & v\\\hline
		
		\texttt{COLOR\_CHANGE\_REQUEST\_HANDLER\_TYPE}& element from $\{$\texttt{VECTOR,NONE}$\}$ & Type of Color Change Request Handler to use & $\{$\texttt{NONE}$\}$\\\hline
		
		 \texttt{STANDARD\_MARKER\_REQUEST\_HANDLER\_SEED} &  integer & Seed for Marker Request Handler to use & $\{$\texttt{NONE}$\}$\\\hline
		 
		\texttt{STANDARD\_TYPE\_CHANGE\_REQUEST\_HANDLER\_SEED} &   integer & Seed for Type Change Request Handler to use. & -\\\hline
		
		\texttt{POSITION\_REQUEST\_HANDLER\_SEED} &   integer & Seed for Position Request Handler to use & -\\\hline
		
		\texttt{COLLISION\_POSITION\_REQUEST\_HANDLER\_SEED} &   integer & Seed for Position Request Handler (of type COLLISION) to use & -\\\hline

		\texttt{VELOCITY\_REQUEST\_HANDLER\_SEED} &   integer & Seed for Velocity Request Handler to use & -\\\hline

		\texttt{ACCELERATION\_REQUEST\_HANDLER\_SEED} &   integer & Seed for Acceleration Request Handler to use & -\\\hline
		
		
		\texttt{STANDARD\_MARKER\_REQUEST\_HANDLER\_DISCARD\_PROB} &  element from interval $[0,1]$ & Discard probability for Marker Request Handler to use & -\\\hline
		 
		\texttt{STANDARD\_TYPE\_CHANGE\_REQUEST\_HANDLER\_DISCARD\_PROB} & element from interval $[0,1]$ & Discard probability  for Type Change Request Handler to use. & -\\\hline
		
		\texttt{POSITION\_REQUEST\_HANDLER\_DISCARD\_PROB} & element from interval $[0,1]$ & Discard probability  for Position Request Handler to use & -\\\hline
		
		\texttt{COLLISION\_POSITION\_REQUEST\_HANDLER\_DISCARD\_PROB} & element from interval $[0,1]$ & Discard probability  for Position Request Handler (of type COLLISION) to use & -\\\hline

		\texttt{VELOCITY\_REQUEST\_HANDLER\_DISCARD\_PROB} & element from interval $[0,1]$ & Discard probability  for Velocity Request Handler to use & -\\\hline

		\texttt{ACCELERATION\_REQUEST\_HANDLER\_DISCARD\_PROB} & element from interval $[0,1]$ & Discard probability  for Acceleration Request Handler to use & -\\\hline
		
		
		\texttt{POSITION\_REQUEST\_HANDLER\_MODIFIER} & list of vector modifiers (see \ref{sec:vectorModifiers}) & List of vector modifiers for Position Request Handler to use & -\\\hline
		
		\texttt{COLLISION\_POSITION\_REQUEST\_HANDLER\_MODIFIER} & list of vector modifiers (see \ref{sec:vectorModifiers}) & List of vector modifiers for Position Request Handler (of type COLLISION) to use & -\\\hline

		\texttt{VELOCITY\_REQUEST\_HANDLER\_MODIFIER} & list of vector modifiers (see \ref{sec:vectorModifiers}) & List of vector modifiers for Velocity Request Handler to use & -\\\hline

		\texttt{ACCELERATION\_REQUEST\_HANDLER\_MODIFIER} & list of vector modifiers (see \ref{sec:vectorModifiers}) & List of vector modifiers for Acceleration Request Handler to use & -\\\hline
		
		\texttt{COLLISION\_POSITION\_REQUEST\_HANDLER\_STRATEGY} & element from $\{$\texttt{STOP,TOUCH}$\}$ & Type of strategy to use for collision handling (see \ref{sec:eventHandlers}) & -\\\hline
		
		\texttt{COLLISION\_POSITION\_REQUEST\_HANDLER\_CLEARANCE} & positive floating point value & Two objects with distance less than this value will be considered colliding & -\\\hline
				
	\end{tabular}
	\caption{Variables in the main project file}\label{tab:mainvars2}
\end{sidewaystable}
\enlargethispage*{2cm}
\thispagestyle{empty}
\clearpage


\subsubsection{Example of a main project file}\index{project file!example}
A main project file may look like:
\lstset{language=tcl}
\begin{lstlisting}
# 
# Description about configuration.
#
	
	PROJECT_NAME="My Exciting Project"
	COMPASS_MODEL="NO_COMPASS"
	ROBOT_FILENAME="myrobots"
	OBSTACLE_FILENAME="myobstacle"
	STATISTICS_MODULE="0"
	ASG="SYNCHRONOUS"
	ROBOT_CONTROL="ROBOT_TYPE_ROBOT_CONTROL"
	MASTER_VIEW="GLOBAL_VIEW"
	VIEW="ONE_POINT_FORMATION_VIEW"
	ONE_POINT_FORMATION_VIEW_RADIUS="5.0"
	
	CAMERA_POSITION="0,0,0"
	CAMERA_DIRECTION="1.5,0,0.5"
	
	MARKER_REQUEST_HANDLER_TYPE="STANDARD"
	STANDARD_MARKER_REQUEST_HANDLER_DISCARD_PROB="0.5"
	STANDARD_MARKER_REQUEST_HANDLER_SEED="1"

	TYPE_CHANGE_REQUEST_HANDLER_TYPE="NONE"
	# no additional variables needed

	POSITION_REQUEST_HANDLER_TYPE="VECTOR"
	VECTOR_POSITION_REQUEST_HANDLER_DISCARD_PROB="0.1"
	VECTOR_POSITION_REQUEST_HANDLER_SEED="3"
	VECTOR_POSITION_REQUEST_HANDLER_MODIFIER="(VECTOR_TRIMMER,1.5);(VECTOR_RANDOMIZER,5,2.5)"

	VELOCITY_REQUEST_HANDLER_TYPE="VECTOR"
	VECTOR_VELOCITY_REQUEST_HANDLER_DISCARD_PROB="0.1"
	VECTOR_VELOCITY_REQUEST_HANDLER_SEED="3"
	VECTOR_VELOCITY_REQUEST_HANDLER_MODIFIER="(VECTOR_TRIMMER,1.5);(VECTOR_RANDOMIZER,5,2.5)"
\end{lstlisting}


\subsection{Robot file}\index{robot file}
Tho robot file contains all information about the individual robots. Each non-comment line\footnote{The first line is always a comment, even if not marked as comment.} of this file represents one robot. One line represents the individual information of one robot in a CVS compatible format. That means, between each values is a comma and the information are given in a well defined order.

Therefore the information for one robot has to be saved in exactly one line of the file. Again: the order of this data is important!

\subsubsection{Order of information in one line}
The first line of a robot file generated by the \RSS looks like the following. This is also the order of how information must be set into one row.
\begin{lstlisting}
	  "ID","x-position","y-position","z-position","type","x-velocity","y-velocity","z-velocity","x-acceleration","y-acceleration","z-acceleration","status","marker-info","algorithm","color","x-axis-1","x-axis-2","x-axis-3","y-axis-1","y-axis-2","y-axis-3","z-axis-1","z-axis-2","z-axis-3"
\end{lstlisting}

\paragraph{Meaning of columns}
\begin{description}
	\item [ID] The robot ID number.
	\item [(x/y/z)-position] The initial position, each as floating point value. 
	\item [type] The robot type (for instance master, slave,$\ldots$).
	\item [(x/y/z)-velocity] The initial velocity, each as floating point value. If not needed, set to $0$.
	\item [(x/y/z)-acceleration] The initial acceleration, each as floating point value. If not needed, set to $0$.
	\item [status] Initial status (needs to be defined in you algorithm).%TODO more information!
	\item [marker-info] Initial marker information. %TODO (still has to be specified)
	\item [algorithm] The robot algorithm to use for calculation an moving. You can either set this to a hard-coded robot algorithm provided by the \RSS, or to a \Lua-Script algorithm.
	  \begin{itemize}
	  	\item To use a robot algorithm from the \RSS just specify the name.
		\item To use a \Lua-Script insert the name of the file, including the ending ``.lua''. This may be an relativ path to this file.
	  \end{itemize}
	%TODO give names of all robot algorithms!
	\item [color] This value gives the color the robot has during the \RSS simulation. This color is only utilized in visualization and does not have any effect on the simulation itself. You can use color for instance for a special treatment during the visualization. The color value is an integer from $0$ to $9$. Each value represents a color:
		\begin{description}
			\item [0] green
			\item [1] blue
			\item [2] cyan
			\item [3] red
			\item [4] magenta
			\item [5] yellow
			\item [6] white
			\item [7] black
			\item [8] orange
			\item [9] purple
		\end{description}
	\item [(x/y/z)\texttimes(x/y/z)\texttimes(x/y/z) coordinate system axes] The coordinate system is represented by three vectors. Thus you need to specify three times three coordinates to uniquely define all axis.
\end{description}

\paragraph{Please Note}
\begin{itemize}
	\item Each non-number needs to be quoted.
	\item Each number may be quoted or not.
	\item You can declare specific lines as comments by setting \# as the first sign of the corresponding line.
\end{itemize}


\subsubsection{Example of a robot file}
\begin{lstlisting}
	"ID","x-position","y-position","z-position","type","x-velocity","y-velocity","z-velocity","x-acceleration","y-acceleration","z-acceleration","status","marker-info","algorithm","color","x-axis-1","x-axis-2","x-axis-3","y-axis-1","y-axis-2","y-axis-3","z-axis-1","z-axis-2","z-axis-3"
	0,5.3,9.2,6.4,"master",1.5,2.5,3.5,1.5,2.5,3.5,"sleeping",0,0,0,1,0,0,0,1,0,0,0,1
	1,"2.5","4.2","8.8","slave",1.5,2.5,3.5,1.5,2.5,3.5,"ready",0,1,0,1,0,0,0,1,0,0,0,1
\end{lstlisting}

\subsection{Obstacle file}\index{obstacle file}
Like the robot file the obstacle file uses a csv-compatible format. 
Therefore the information for one robot has to be saved in exactly one line of the file.
Each line contains the following data. The order of this data is important!

You can declare specific lines as comments by setting \# as the first sign of
the corresponding line.
\begin{itemize}
	\item type (marker, sphere or box)
	\item position $(x,y,z)$
	\item marker information (still needs to be specified)
	\item $x/y/z$-lengths or radius (depending on type)
\end{itemize}

The first line always is (column headers):
\begin{lstlisting}
"type","x-position","y-position","z-position","marker-info","size-info","",""
\end{lstlisting}
Each non-number is quoted.


\subsubsection{Example of an obstacle file}
\begin{lstlisting}
"type","x-position","y-position","z-position","marker-info","size-info","",""
"box",2.0,3.0,4.0,0,1.0,2.0,3.0,
"sphere",3.4,5.2,5.1,0,5.0,"",""
"marker",3.5,1.4,5.1,0,"","",""
\end{lstlisting}
As you can already see in the example, if the type of an obstacle is sphere, then the last two values must be empty, i.\,e. '',''. Analoguos, if the type is marker, the last three values must be empty, i.\,e. '','',''.


\newpage
%%
% This file is part of the User's Guide to RSS
% It contains the appendix for scaling tests
%%

\section{Scaling Tests}

We present preliminary performance results of \RSS. The goal of the performance tests was to find out how the software scales with the number of simulated robots and which parts of the simulation take the most time to compute. All measurements have been taken on a Windows Vista PC with the following hardware: Intel Core 2 Duo at 2,53\,Ghz and 4\,GB DDR3 Ram at 1066\,Mhz. The code has been compiled using {\sffamily MinGW GCC 3.4.5}.

At all times, the visualization component of \RSS has been switched on. It runs in its own thread and consumes world information objects generated by the actual simulation kernel. To get an idea of the impact of visualization on the performance as a whole, we independently measured how long setting up each rendering frame takes. In between each frame, a variable amount of simulation time passes by, this amount is referred to as processing time. Figure~\ref{pic:scaling:render} shows that rendering one frame increases linearly with the amount of robots. At the target of 1000 robots it takes about 0.01 seconds. 

\begin{figure}[p]
	\begin{center}
	\includegraphics[width=0.8\textwidth]{scaling-render}
	\caption{Scaling test of rendering}
	\label{pic:scaling:render}
	\end{center}
\end{figure}

In the synchronous time model, each simulation step consists of updating all robots views (look event), executing each robots algorithm on that view (compute event) and then handling each robot's request to the world state (handle request event). We measured the computation time for each of these three events. The figures given represent the average time for one event after 33 steps have been simulated.

We first look at the simplest case when no algorithm at all is loaded onto the robots and they receive the global view. All robots are distributed randomly in space and receive random velocity and acceleration. Figure~\ref{pic:scaling:noglobal} tells us that Look and Request events have similar computation times and stay constant regarding the total number of robots. The compute events take almost no time and can be disregarded here. Specifically, Look and Request executed in respectively 0.15 and 0.34 seconds which results in around 0.5 seconds for a simulation step. Note that the robots actually do not return any requests so that this time can probably be regarded as overhead for copying data around.

\begin{figure}[p]
	\begin{center}
	\includegraphics[width=0.8\textwidth]{scaling-no-global}
	\caption{Scaling test -- no algorithm and global view}
	\label{pic:scaling:noglobal}
	\end{center}
\end{figure}

Now we consider the Circle algorithm, specified in \texttt{circle.lua}. Here, every robot calculates the center point of all visible robots and generates a velocity request so that it rotates around the center. The robots form a rotating ring. It has been tested with global (one huge ring) and local view (several smaller rings).

Figure~\ref{circle global.pdf} shows how performance scales in this scenario. The three different event types behave very differently. The compute event has a computation time $\mathbb{O}(n^2)$ where $n$ is the number of robots. This should be expected, as each robot regards each other in the computation of the center point. Surprisingly, the times for look and request actually decrease. When simulating 1000 robots, compute is the dominant term as each event takes 13 seconds to execute, compared to 0.010 and 0.014 for look and request respectively.

Consider the same algorithm with only a local view. We use the spheric view with a radius of 3. As it can be seen in Figure~\ref{pic:scaling:circlelocal}, the different types of events scale similarly, however in a different relation to each other. Look and request events take the same time as when using global view, which means that the octree implementation is so efficient that it is not slower than simply copying all the robots. Compute events however only take 0.94 seconds. This is due to the fact that now every robots consider only neighboring robots for calculating the center point.

\begin{figure}[p]
	\begin{center}
	\includegraphics[width=0.8\textwidth]{scaling-circle-local}
	\caption{Scaling test -- circle local}
	\label{pic:scaling:circlelocal}
	\end{center}
\end{figure}

\begin{figure}[p]
	\begin{center}
	\includegraphics[width=0.8\textwidth]{scaling-circle-global}
	\caption{Scaling test -- circle global}
	\label{pic:scaling:circleglobal}
	\end{center}
\end{figure}


% Backcover
\newpage
\thispagestyle{empty}
~
\end{document}
